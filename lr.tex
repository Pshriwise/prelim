\documentclass{article}
%%%%%%%%%%%%%%%%%%%%%%%%%%%%%%%%%%%%%%%%%%%%%%%%%%
\title{The Current State of Spatial Paritioning Trees: A Literature Review}
\author{Patrick C. Shriwise}
\date{\today}
%\institute{Department of Engineering Physics, University of Wisconsin-Madison, 1500 Engineering Dr, Madison, WI 53706, shriwise@wisc.edu}



%%%%% packages and defs
\usepackage{graphicx}
\usepackage{epsfig}
\usepackage{float}
\usepackage{booktabs}
\usepackage[font={small,it}]{caption}

\begin{document}

\maketitle

\tableofcontents 

\section{Abstract}

The value of CAD-Based Monte Carlo relies heavily on the ability to return geometric queries quickly and robustly via ray tracing. The Direct Accelerated Monte Carlo (DAGMC) toolkit currently provides a robust ray tracing algorithm\cite{thesis_smith_2010} given certain mesh requirements are met, but shortcomings of the underlying ray tracing kernel causes it to lag behind run times of a given Monte Carlo code using its native geometric analysis. Up to this point, we have relied on research in ray tracing mainly for the application of rendering or collision detection. Neither of these applications align exactly with the application of Monte Carlo ray tracing. Algorithms for collision detection involves moving geometry which isn’t yet a concern in the area of radiation transport. Algorithms for acceleration structures in rendering typically prioritize the time needed for the building of these structures causing them to be less than optimal, but results in an overall decreased time-to-render due to the balance of the build time with the number of ray queries needed to generate the desired graphics. In radiation transport, the number of ray queries needed to generate the desired result is much larger than the number needed for rendering. As more time is spent ray tracing, the build times of the underlying acceleration structures becomes less significant, thus the balance between time spent building these structures and the time using them for ray tracing must be reconsidered in this context. Its is hypothesized that in the case of the large, complex problems for which DAGMC is intended, that extra time spent optimizing the ray tracing acceleration structures via adaptation to certain mesh features and/or mesh refinement/alteration if necessary. This shift in priorities regarding the building process is important, but there are other improvements to be made as well. An area where the application of rendering and radiation transport align very well is in the query process after building of acceleration structures is complete. There are many advancements in ray tracing techniques which can be applied in order to provide further improvements to DAGMC’s performance. It is hoped that the combined effect of these improvements will bring DAGMC’s performance close to that of the native Monte Carlo codes it supports, thus alleviating the concern for additional computational time while maintaining the benefit of reduced human effort in radiation transport on complex geometric models.

\section{Introduction}

%% The purpose of this document is to outline the current state of the art in spatial partioning hierarchies for the purpose of improving performance in DagMC. It will take into account as much of the existing literature as possible. An analysis of the literature will also be conatined, identifying areas for improvement within the context of ray tracing for the specific application of Monte Carlo ray tracing.


The (DAGMC) toolkit \cite{dagmc_2009} robustly tracks particles through geometries represented by a surface mesh provided by the graphics faceting of the CAD engine in which the geometry is generated. Thanks to the make\_watertight algorithm\cite{make_watertight_smith_2010} DAGMC can robustly track particles through highly complex geometries by providing the necessary geometric information to underlying Monte Carlo physics codes via the ray tracing kernel in the Mesh Oriented dAtaBase (MOAB)\cite{moab}.

Naturally following the ability to do these robust calculations on already complex models, increasingly detailed models are becoming of interest. As detail increases, the performace of DAGMC lags more and more in comparison to the native versions of the various Monte Carlo codes it supports. It has long been known that this performance issue is due to the time it takes to return geometric information to Monte Carlo packages from the ray tracing kernel. Recent profiling of the DAGMC code supports this knowledge. This work has shown that in complex models ~85\% of the run time is spent in the ray tracing kernel. As this area seems to be the main bottleneck, this work will focus on optimizing this area in DAGMC.

The purpose of most applications involving ray tracing is rendering. Techiques used in rendering for accelerating ray tracing queries have proved very useful in the area of Monte Carlo analysis. However, these techniques have largely been applied without modification or adaptation for the purposes of ray tracing in Monte Carlo. Several fundamental differences exist between the two applications. Exploration of how ray tracing kernels inteded for use in Monte Carlo codes can be modified to the benefit of performance will be explored in this work.

* we fire more rays, damnit! affects time to solution
* also, SIMD could be a big win (espeically for history-based threading)
* we know what volume we're in

\section{Ray Tracing's Role in Monte Carlo}

At large, ray tracing is used to generate 2D images by simulating light ray interactions with virtual 3D objects. These light rays (photons) are typically cast from the perspective of the image to be generated, and pixel colors are determined by the reflections, refractions, etc. of the photons on their course to the light source (if they reach it at all). This process is analagous to Monte Carlo analysis. Particles are born at the source and go through a series of interactions with 3D objects until they reach an area of interest, or detector, at which point data related to the particle is collected and stored. In both cases, when particles obtain new orientations from their interactions in this virtual 3D space, it is necessary to understand what object this particle will interact with next if it continues on its current trajectory. As such, the problem of ray tracing in either application can be abstracted to a spatial search of a virtual 3D object for the next interaction location given a particle loaction and trajectory. 

In realistic renderings, it is sometimes possible that light rays may scatter before reaching the next object on their path and in Monte Carlo simulations this is the more probable case. In order to determine if this will occur before reaching the next object, the distance to interaction (scattering or otherwise) must be compared to the distance to the next object. In analytic geometric representations this is a straightforward task, but as the complexity of 3D objects increase, the cost of ray-object intersections becomes very computationally expensive. Mesh/CAD-based models, in which the surfaces of objects are discretized into sets of geometric primitives (usually triangles), face the challenge of then finding the correct primitive for which the ray intersects (if it intersects one at all). For the highly complex models, a linear search of the primitives is out of the question in terms of how long these intersections would take to find. As a result, extensive research has been done in the area of acceleration data structures for this ray tracing problem.

\section{Spatial Partitioning Schemes}


The area of computational spatial searching using partioning hierarchies (heretofore referred to as spatial trees) has been around for some time. As a result, many concepts for partitioning the space contained by a set of data have been developed and applied. In order to establish a successful spatial partitioning tree, one needs to have: 

\begin{enumerate}
  \item A tree structure.
  \item A partitioning construct.
  \item A division heuristic.
  \item Leaf conditions.
\end{enumerate}

The tree structure refers almost entirely to the number of children each non-leaf node in the tree should have. The majority of spatial trees are binary trees (two children), but other structures do exist such as quad trees (four children) and oct trees (eight children). The partitioning construct refers to the mechanism by which you are creating partitions out of space. There is a large variety of these which are chosen based on a couple of counteracting criterion 1) number of operations required for an intersection check and 2) their ability to effectively eliminate subsets of the problem space from a given query.
 
\subsection{Bounding Volume Hierarchies}

The initial concept of using primitive bounding volumes as a pre-check for ray-intersection with objects was introduced by Weghorst in 1984 \cite{Weghorst:1984:ICM}. Weghorst explored the possibility of using spheres and rectangular parallell pipeds to contain geometric objects. This work also went so far as to create a hierarchy of those bounding volumes, noting the importance of joining boxes bounding objects which are near each other in space so as not to have bounding volumes containing large amounts of empty space between the objects to be joined. Another important conclusion of this work was that while spherical bounding volumes are less computationally expensive to check for ray intersections than bounding boxes, the latter generally provide a tighter fit to the objects they contain. This tighter fitting is important as it decreases the chance of wasted intersections further down in the spatial hierarchy.

Over time it was found that complex geometric objects become difficult to render as the analytic calculation of ray interactions is costly. As a result, representations of geometric objects were then discretized for renderings. The ray interactions with these more simple geometric primitives (usually triangles) are much less costly to compute in comparison to the previous analytic representation. This also allowed for the rendering of arbitrarily complex geometric objects provided that they could be discretized into a triangle mesh. As the models were discretized, the sptial data structures designed for ray queries followed. Hierarchies of bounding volumes could then be created on subsets of the primitives, making for deepter and even more effective accelerations of ray tracing.


\subsubsection{Axis Aligned Bounding Box Trees}

\subsubsection{Oriented Bounding Box Trees}

\subsection{Planar Splitting Schemes}

\subsubsection{K-D Trees}

\subsubsection{Bounding Interval Trees}

\subsection{Other Schemes}

\subsubsection{Oct-Trees}

\bibliographystyle{acm}
\bibliography{refs}

\end{document}
