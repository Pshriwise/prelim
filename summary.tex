\documentclass{anstrans}
%%%%%%%%%%%%%%%%%%%%%%%%%%%%%%%%%%%
\title{FluDAG and Other Implementations of the DAGMC Toolkit}
\author{Paul P.H. Wilson, Andrew Davis, Julie Zachman, Kerry L. Dunn}

\institute{University of Wisconsin-Madison, 1500 Engineering Dr, Madison, WI 53706}

\email{wilsonp@engr.wisc.edu}

% Optional disclaimer: remove this command to hide

%%%% packages and definitions (optional)
\usepackage{graphicx} % allows inclusion of graphics
\usepackage{booktabs} % nice rules (thick lines) for tables
\usepackage{microtype} % improves typography for PDF

\begin{document}
%%%%%%%%%%%%%%%%%%%%%%%%%%%%%%%%%%%%%%%%%%%%%%%%%%%%%%%%%%%%%%%%%%%%%%%%%%%%%%%%
\section{Introduction}

The Direct Accelerated Geometry Monte Carlo (DAGMC) toolkit provides
an interface to the Mesh-Oriented datABase (MOAB) specialized for the
geometric operations of Monte Carlo radiation transport directly on
CAD-based geometries.  By taking full control of the geometric
operations, including ray tracing, point inclusion, and surface
determination, DAGMC allows an identical geometry representation to be
used in a variety of Monte Carlo physics solvers.

This work describes the application of the DAGMC toolkit to
Fluka\cite{fluka}, a Monte Carlo physics solver from CERN with support
for a wide array of particles at a broad range of energies.  The Fluka
implementation of DAGMC relies on previous work to integrate
GEANT4\cite{GEANT4} geometry capability with the Fluka physics solver,
known as FluGG\cite{flugg}.

In addition, we provide updates on efforts to integrate DAGMC with
other physics solvers including MCNP5\cite{mcnp5},
Tripoli4\cite{tripolo}, GEANT4, and SHIFT\cite{shift}.

%%%%%%%%%%%%%%%%%%%%%%%%%%%%%%%%%%%%%%%%%%%%%%%%%%%%%%%%%%%%%%%%%%%%%%%%%%%%%%%%
\section{The DAGMC Toolkit}

The details of DAGMC have been described in other work, but are
summarized here for context.  A user generates a geometric description
of the problem using one of a variety of solid modeling software
packages and imports that geometry into Cubit\cite{cubit} for
pre-processing steps including assignment of materials and boundary
condition, and imprinting and merging.  This geometry file is used to
generate a facetted representation of the geometry that is stored in a
MOAB format, with precision adequate to guarantee reliable radiation
transport results\cite{snouffer}.  When loaded into a physics solver
modified with DAGMC, the material and boundary condition informaiton
is extracted and an oriented bounding box tree is generated to enable
accelerated geometry operations on this representation with $O(10^6)$
facets.  DAGMC allows simulation of complex geometries without
time-consuming translation to native geometric representations of the
physics solvers, with modest performance impact.  It also allows the
simulation of high-order surfaces that may not be supported by the
native geometric representation.

The DAGMC toolkit is comprised primarily of a small set of geometric
operations typical of Monte Carlo radiation transport: ray-tracing,
point inclusion, and normal determination.  The DAGMC interface also
includes methods for handling meta-data such as material assignment
and boundary conditions.  Integration with a physics solver typically
requires identifying all the geometry queries made by the phsyics
solver and replacing them with queries to the DAGMC interface.  In
addition, various approaches are necessary to manage the
initialziation phase of the simulation, depending on the physics
solver's initialization methods.  The primary physics solver used with
DAGMC has been MCNP5, providing unique computing capability for
analysis of shielding in fusion systems\cite{fusion3d} and criticality
deformed space nuclear energy systems\cite{nets}.

%%%%%%%%%%%%%%%%%%%%%%%%%%%%%%%%%%%%%%%%%%%%%%%%%%%%%%%%%%%%%%%%%%%%%%%%%%%%%%%%
\section{Fluka and FluGG}

%%%%%%%%%%%%%%%%%%%%%%%%%%%%%%%%%%%%%%%%%%%%%%%%%%%%%%%%%%%%%%%%%%%%%%%%%%%%%%%%
\section{Acknowledgments}
This work is supported in part by NASA under contract XXXXX

%%%%%%%%%%%%%%%%%%%%%%%%%%%%%%%%%%%%%%%%%%%%%%%%%%%%%%%%%%%%%%%%%%%%%%%%%%%%%%%%
\bibliographystyle{ans}
\bibliography{bibliography}
\end{document}

